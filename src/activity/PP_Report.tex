%%%%%%%%%%%%%%%%%%%%%%%%%%%%%%%%%%%%%%%%%%%%%%%%%%%%%%%%%%%%%%%%%%%%%%%%%%%%%%%%
%2345678901234567890123456789012345678901234567890123456789012345678901234567890
%        1         2         3         4         5         6         7         8
%% PP_Report.tex
%% V1.4
%% 2015/11/30
%% by Rui Santos Cruz
%% This is a skeleton file using PPIEEEtran.cls
%% (requires PPIEEEtran.cls) 
% !TEX root = ./main.tex
%%%%%%%%%%%%%%%%%%%%%%%%%%%%%%%%%%%%%%%%%%%%%%%%%%%%%%%%%%%%%%%%%%%%%%%%%%%%%%%%
\documentclass[a4paper,12pt,journal,twoside,compsoc]{PPIEEEtran}

% -----------------------------------------------------------------------------
% The Preamble document contains all the necessary Packages for typesetting
% Modify it to suit your needs
% -----------------------------------------------------------------------------
%%%%%%%%%%%%%%%%%%%%%%%%%%%%%%%%%%%%%%%%%%%%%%%%%%%%%%%%%%%%%%%%%%%%%%%%%%%%%%%%
%2345678901234567890123456789012345678901234567890123456789012345678901234567890
%        1         2         3         4         5         6         7         8
% Required Packages and commands
% --> Please Choose the MAIN LANGUAGE for the document in package BABEL (below)
% --> Please Choose the TYPE OF REPORT for the document in \ReportType (below)
% !TEX root = ./main.tex
% PP_Report_Preamble.tex
% V1.4
% 2015/11/30
% by Rui Santos Cruz
%%%%%%%%%%%%%%%%%%%%%%%%%%%%%%%%%%%%%%%%%%%%%%%%%%%%%%%%%%%%%%%%%%%%%%%%%%%%%%%%
%
% *** INPUT LANGUAGE PACKAGES ***

\usepackage[main=english,portuguese]{babel}
\usepackage[utf8]{inputenc}
\usepackage{iflang}

% *** DEFINE THE TYPE OF REPORT ***
%\newcommand*{\ReportType}{learning}% Uncomment line for Learning Report
\newcommand*{\ReportType}{activity}% Uncomment line for Activity Report

% *** ACRONYM PACKAGES ***
% Put definition of Acronyms at the end of the document
\usepackage[printonlyused,nolist]{acronym}

% *** CITATION PACKAGES ***
\usepackage{cite}

% *** GRAPHICS RELATED PACKAGES ***
\usepackage[pdftex]{graphicx}
\DeclareGraphicsExtensions{.pdf,.jpeg,.png}

% *** MATH PACKAGES ***
\usepackage[cmex10]{amsmath}

% *** SPECIALIZED LIST PACKAGES ***
\usepackage{algorithmic}

% *** ALIGNMENT PACKAGES ***
\usepackage{array}

% *** SUBFIGURE PACKAGES ***
\usepackage[caption=false,font=normalsize,labelfont=sf,textfont=sf]{subfig}

% *** FLOAT PACKAGES ***
\usepackage{fixltx2e}

% *** PDF, URL AND HYPERLINK PACKAGES ***
\usepackage{url}

% *** BACKGROUND Material ***
\usepackage{eso-pic}
\usepackage[
  contents={},
  opacity=1,
  scale=1,
  color=blue!90
  ]{background}
  
% *** CONDITIONALS ***
\usepackage{ifthen}

% DEFINE COMMAND FOR: Report Type depending on language
\newcommand{\tlangRepActivity}{\IfLanguageName{english}{Activity Report}{Relatório de Atividade}}
\newcommand{\tlangRepLearning}{\IfLanguageName{english}{Learnings Report}{Relatório de Aprendizagens}}
%%%%%%%%%%%%%%%%%%%%%%%%%%%%%%%%%%%%%%%%%%%%%%%%%%%%%%%%%%%%%%%%%%%%%%%%%%%%%%%%
% DEFINE COMMAND FOR: Report Scoring Table Type
\newcommand{\lrScore}%
{\setlength{\unitlength}{1mm}{% % selecting unit length 
\fontfamily{phv}\selectfont
\begin{picture}(171.6,20) % picture environment with the size (dimensions)
% 32 length units wide, and 15 units high.
\setlength\fboxsep{0pt}
% Left Set with grading Scores
\put(0,0){\fcolorbox{gray}{gray!20}{%
          \framebox(15,4)[l]{\scriptsize{0.2-Weak}}}}
\put(0,4){\fcolorbox{gray}{gray!20}{%
          \framebox(15,4)[l]{\scriptsize{0.4-Fair}}}}
\put(0,8){\fcolorbox{gray}{gray!20}{%
          \framebox(15,4)[l]{\scriptsize{0.6-Good}}}}
\put(0,12){\fcolorbox{gray}{gray!20}{%
           \framebox(15,4)[l]{\scriptsize{0.8-V.Good}}}}
\put(0,16){\fcolorbox{gray}{gray!20}{%
           \framebox(15,4)[l]{\scriptsize{1.0-Excel}}}}
% Left+1 Set with Learning Rubrics
\put(16,0){\fcolorbox{cyan}{white}{%
          \framebox(12,12)[c]{\footnotesize{ }}}}
\put(28,0){\fcolorbox{cyan}{white}{%
          \framebox(12,12)[c]{\footnotesize{ }}}}          
\put(40,0){\fcolorbox{cyan}{white}{%
          \framebox(12,12)[c]{\footnotesize{ }}}}
\put(52,0){\fcolorbox{cyan}{white}{%
          \framebox(12,12)[c]{\footnotesize{ }}}}
\put(64,0){\fcolorbox{cyan}{white}{%
          \framebox(12,12)[c]{\footnotesize{ }}}}
\put(16,12){\fcolorbox{cyan}{white}{%
          \framebox(12,4)[c]{\tiny{Intro$\times 2$}}}}
\put(28,12){\fcolorbox{cyan}{white}{%
          \framebox(12,4)[c]{\tiny{Motiv$\times 2$}}}}          
\put(40,12){\fcolorbox{cyan}{white}{%
          \framebox(12,4)[c]{\tiny{Skills$\times 6$}}}}
\put(52,12){\fcolorbox{cyan}{white}{%
          \framebox(12,4)[c]{\tiny{Reflect$\times 6$}}}}
\put(64,12){\fcolorbox{cyan}{white}{%
          \framebox(12,4)[c]{\tiny{Sugg$\times 2$}}}}
\put(16,16){\fcolorbox{cyan}{cyan!20}{%
          \framebox(60,4)[c]{\footnotesize{LEARNINGS}}}}
% Middle Set with Document Rubrics
\put(77,0){\fcolorbox{green}{white}{%
          \framebox(12,12)[c]{\footnotesize{ }}}}
\put(89,0){\fcolorbox{green}{white}{%
          \framebox(12,12)[c]{\footnotesize{ }}}}
\put(101,0){\fcolorbox{green}{white}{%
          \framebox(12,12)[c]{\footnotesize{ }}}}
\put(113,0){\fcolorbox{green}{white}{%
          \framebox(12,12)[c]{\footnotesize{ }}}}
\put(125,0){\fcolorbox{green}{white}{%
          \framebox(12,12)[c]{\footnotesize{ }}}}
\put(137,0){\fcolorbox{green}{white}{%
          \framebox(12,12)[c]{\footnotesize{ }}}}
\put(77,12){\fcolorbox{green}{white}{%
          \framebox(12,4)[c]{\tiny{Struct $\times .25$}}}}
\put(89,12){\fcolorbox{green}{white}{%
          \framebox(12,4)[c]{\tiny{Ortog$\times .25$}}}}          
\put(101,12){\fcolorbox{green}{white}{%
          \framebox(12,4)[c]{\tiny{Gram$\times .25$}}}}
\put(113,12){\fcolorbox{green}{white}{%
          \framebox(12,4)[c]{\tiny{Form $\times .25$}}}}
\put(125,12){\fcolorbox{green}{white}{%
          \framebox(12,4)[c]{\tiny{Abstr $\times .5$}}}}
\put(137,12){\fcolorbox{green}{white}{%
          \framebox(12,4)[c]{\tiny{Concl $\times .5$}}}}
\put(77,16){\fcolorbox{green}{green!20}{%
          \framebox(72,4)[c]{\footnotesize{DOCUMENT}}}}
% Right Set with Penalties
\put(150,0){\fcolorbox{red}{white}{%
          \framebox(10,12)[c]{\footnotesize{ }}}}
\put(160,0){\fcolorbox{red}{white}{%
          \framebox(10,12)[c]{\footnotesize{ }}}}
\put(170,0){\fcolorbox{red}{white}{%
          \framebox(10,12)[c]{\footnotesize{ }}}}
\put(150,12){\fcolorbox{red}{white}{%
          \framebox(10,4)[c]{\tiny{Titles $\times .5$}}}}
\put(160,12){\fcolorbox{red}{white}{%
          \framebox(10,4)[c]{\tiny{Files $\times .5$}}}}
\put(170,12){\fcolorbox{red}{white}{%
          \framebox(10,4)[c]{\tiny{IDs $\times .5$}}}}
\put(150,16){\fcolorbox{red}{red!20}{%
          \framebox(30,4)[c]{\footnotesize{PENALTY}}}}
\end{picture}
}}
%%%%%%%%%%%%%%%%%%%%%%%%%%%%%%%%%%%%%%%%%%%%%%%%%%%%%%%%%%%%%%%%%%%%%%%%%%%%%%%%
%\newcommand{\arScore}%
\newcommand{\arScore}%
{\setlength{\unitlength}{1mm}{% % selecting unit length 
\fontfamily{phv}\selectfont
\begin{picture}(171.6,20) % picture environment with the size (dimensions)
% 32 length units wide, and 15 units high.
\setlength\fboxsep{0pt}
% Left Set with grading Scores
\put(0,0){\fcolorbox{gray}{gray!20}{%
          \framebox(15,4)[l]{\scriptsize{0.2-Weak}}}}
\put(0,4){\fcolorbox{gray}{gray!20}{%
          \framebox(15,4)[l]{\scriptsize{0.4-Fair}}}}
\put(0,8){\fcolorbox{gray}{gray!20}{%
          \framebox(15,4)[l]{\scriptsize{0.6-Good}}}}
\put(0,12){\fcolorbox{gray}{gray!20}{%
           \framebox(15,4)[l]{\scriptsize{0.8-V.Good}}}}
\put(0,16){\fcolorbox{gray}{gray!20}{%
           \framebox(15,4)[l]{\scriptsize{1.0-Excel}}}}
% Left+1 Set with Activity Rubrics
\put(16,0){\fcolorbox{yellow}{white}{%
          \framebox(12,12)[c]{\footnotesize{ }}}}
\put(28,0){\fcolorbox{yellow}{white}{%
          \framebox(12,12)[c]{\footnotesize{ }}}}          
\put(40,0){\fcolorbox{yellow}{white}{%
          \framebox(12,12)[c]{\footnotesize{ }}}}
\put(52,0){\fcolorbox{yellow}{white}{%
          \framebox(12,12)[c]{\footnotesize{ }}}}
\put(64,0){\fcolorbox{yellow}{white}{%
          \framebox(12,12)[c]{\footnotesize{ }}}}
\put(16,12){\fcolorbox{yellow}{white}{%
          \framebox(12,4)[c]{\tiny{Intro$\times 2$}}}}
\put(28,12){\fcolorbox{yellow}{white}{%
          \framebox(12,4)[c]{\tiny{Object$\times 2$}}}}          
\put(40,12){\fcolorbox{yellow}{white}{%
          \framebox(12,4)[c]{\tiny{Plan$\times 4$}}}}
\put(52,12){\fcolorbox{yellow}{white}{%
          \framebox(12,4)[c]{\tiny{Exec$\times 6$}}}}
\put(64,12){\fcolorbox{yellow}{white}{%
          \framebox(12,4)[c]{\tiny{Result$\times 4$}}}}
\put(16,16){\fcolorbox{yellow}{yellow!20}{%
          \framebox(60,4)[c]{\footnotesize{ACTIVITY}}}}
% Middle Set with Document Rubrics
\put(77,0){\fcolorbox{green}{white}{%
          \framebox(12,12)[c]{\footnotesize{ }}}}
\put(89,0){\fcolorbox{green}{white}{%
          \framebox(12,12)[c]{\footnotesize{ }}}}
\put(101,0){\fcolorbox{green}{white}{%
          \framebox(12,12)[c]{\footnotesize{ }}}}
\put(113,0){\fcolorbox{green}{white}{%
          \framebox(12,12)[c]{\footnotesize{ }}}}
\put(125,0){\fcolorbox{green}{white}{%
          \framebox(12,12)[c]{\footnotesize{ }}}}
\put(137,0){\fcolorbox{green}{white}{%
          \framebox(12,12)[c]{\footnotesize{ }}}}
\put(77,12){\fcolorbox{green}{white}{%
          \framebox(12,4)[c]{\tiny{Struct $\times .25$}}}}
\put(89,12){\fcolorbox{green}{white}{%
          \framebox(12,4)[c]{\tiny{Ortog$\times .25$}}}}          
\put(101,12){\fcolorbox{green}{white}{%
          \framebox(12,4)[c]{\tiny{Gram$\times .25$}}}}
\put(113,12){\fcolorbox{green}{white}{%
          \framebox(12,4)[c]{\tiny{Form $\times .25$}}}}
\put(125,12){\fcolorbox{green}{white}{%
          \framebox(12,4)[c]{\tiny{Abstr $\times .5$}}}}
\put(137,12){\fcolorbox{green}{white}{%
          \framebox(12,4)[c]{\tiny{Concl $\times .5$}}}}
\put(77,16){\fcolorbox{green}{green!20}{%
          \framebox(72,4)[c]{\footnotesize{DOCUMENT}}}}
% Right Set with Penalties
\put(150,0){\fcolorbox{red}{white}{%
          \framebox(10,12)[c]{\footnotesize{ }}}}
\put(160,0){\fcolorbox{red}{white}{%
          \framebox(10,12)[c]{\footnotesize{ }}}}
\put(170,0){\fcolorbox{red}{white}{%
          \framebox(10,12)[c]{\footnotesize{ }}}}
\put(150,12){\fcolorbox{red}{white}{%
          \framebox(10,4)[c]{\tiny{Titles $\times .5$}}}}
\put(160,12){\fcolorbox{red}{white}{%
          \framebox(10,4)[c]{\tiny{Files $\times .5$}}}}
\put(170,12){\fcolorbox{red}{white}{%
          \framebox(10,4)[c]{\tiny{IDs $\times .5$}}}}
\put(150,16){\fcolorbox{red}{red!20}{%
          \framebox(30,4)[c]{\footnotesize{PENALTY}}}}
\end{picture}
}}

% DEFINE COMMAND FOR: Printing Scoring Table Type
\newcommand\BackgroundPic{%
\put(-15,12){%
\parbox[b][\paperheight]{\paperwidth}{%
\vfill
\centering
\ifthenelse{\equal{\ReportType}{activity}}{\arScore}{\lrScore}}}}
% Printing the Scoring Table
\AddToShipoutPicture*{\BackgroundPic}

% Print Vertical Identifications on even and odd pages
\AddEverypageHook{%
  \ifthenelse{\isodd{\value{page}}}%
  {\backgroundsetup{
    angle=90,
    position={-0.1\textwidth,-1.055\textheight},
    contents={\tiny{PP-2015 V1.4}}
    }% Odd Pages
  }%
  {\backgroundsetup{
    angle=90,
    position={0.97\textwidth,-1.05\textheight},%
    contents={\ifthenelse{\equal{\ReportType}{activity}}{%
              \tiny{\tlangRepActivity}}{\tiny{\tlangRepLearning}}}
    }% Even Pages
  }%
  \BgMaterial}
% correct bad hyphenation here
\hyphenation{op-tical net-works semi-conduc-tor}
%%%%%%%%%%%%%%%%%%%%%%%%%%%%%%%%%%%%%%%%%%%%%%%%%%%%%%%%%%%%%%%%%%%%%%%%%%%%%%%%
%2345678901234567890123456789012345678901234567890123456789012345678901234567890
%        1         2         3         4         5         6         7         8
\begin{document}
%%%%%%%%%%%%%%%%%%%%%%%%%%%%%%%%%%%%%%%%%%%%%%%%%%%%%%%%%%%%%%%%%%%%%%%%%%%%%%%%
%2345678901234567890123456789012345678901234567890123456789012345678901234567890
%        1         2         3         4         5         6         7         8
%% PP_Report_Cover.tex
%% V1.4
%% 2015/11/30
%% by Rui Santos Cruz
% !TEX root = ./main.tex
%%%%%%%%%%%%%%%%%%%%%%%%%%%%%%%%%%%%%%%%%%%%%%%%%%%%%%%%%%%%%%%%%%%%%%%%%%%%%%%%
% paper title
% can use linebreaks \\ within to get better formatting as desired
% Do not put math or special symbols in the title.
\title{Por2folios Platform}
%%%%%%%%%%%%%%%%%%%%%%%%%%%%%%%%%%%%%%%%%%%%%%%%%%%%%%%%%%%%%%%%%%%%%%%%%%%%%%%%
% Author names
%
% note positions of commas and nonbreaking spaces ( ~ ) LaTeX will not break
% a structure at a ~ so this keeps an author's name from being broken across
% two lines.
% use \thanks{} to gain access to the first footnote area
% a separate \thanks must be used for each paragraph.
%
%\IEEEcompsocitemizethanks is a special \thanks that produces the bulleted
% lists for "first footnote" author affiliations. 
% Use \IEEEcompsocthanksitem which works much like \item
% for each affiliation group.
\author{Francisco~Maria~Calisto% <-this % stops a space
% Change the Course Name 
% note: need leading \protect in front of \\ to get a newline within \thanks as
% \\ is fragile and will error, could use \hfil\break instead.
\IEEEcompsocitemizethanks{
\IEEEcompsocthanksitem Bruno~Cardoso, nr. 72619,\protect\\ 
E-mail: bruno.f.cardoso@tecnico.ulisboa.pt,
\IEEEcompsocthanksitem Francisco~Maria~Calisto, nr. 70916,\protect\\
E-mail: francisco.calisto@tecnico.ulisboa.pt,
\IEEEcompsocthanksitem Nuno~Sousa, nr. 73216,\protect\\
E-mail: nuno.g.sousa@tecnico.ulisboa.pt,\protect\\
Instituto Superior Técnico, Universidade de Lisboa.\protect\\}% <-this % stops an unwanted space}% <-this % stops an unwanted space
\thanks{Manuscrito recebido a 24 de Junho de 2016.}
}
%%%%%%%%%%%%%%%%%%%%%%%%%%%%%%%%%%%%%%%%%%%%%%%%%%%%%%%%%%%%%%%%%%%%%%%%%%%%%%%%
% The paper headers
\markboth{Por2folios}%
% for a single student
%{Surname}% : for a single student 
% for a Group Report 
{Surname \MakeLowercase{\textit{et al.}}}% : for a Group Report 
%
% The only time the second header will appear is for the odd numbered pages
% after the title page when using the twoside option.
%%%%%%%%%%%%%%%%%%%%%%%%%%%%%%%%%%%%%%%%%%%%%%%%%%%%%%%%%%%%%%%%%%%%%%%%%%%%%%%%
% Prints in Subtitle the type of Report
% PLEASE DO NOT CHANGE THIS SECTION
\IEEEspecialpapernotice{%
\ifthenelse{\equal{\ReportType}{activity}}{%
\tlangRepActivity}{\tlangRepLearning}
}
%%%%%%%%%%%%%%%%%%%%%%%%%%%%%%%%%%%%%%%%%%%%%%%%%%%%%%%%%%%%%%%%%%%%%%%%%%%%%%%%
%%%%%%%%%%%%%%%%%%%%%%%%%%%%%%%%%%%%%%%%%%%%%%%%%%%%%%%%%%%%%%%%%%%%%%%%%%%%%%%%
% The paper Abstract and Keywords
\IEEEtitleabstractindextext{%
\begin{abstract}
Este relatório tem como objectivo descrever a actividade e a aprendizagem que o projecto Plataforma Por2folios nos proporcionou ao desenvolver uma plataforma que irá reunir todos os projectos, trabalhos e relatórios dos vários anos lectivos em que a cadeira foi lecionada.
\end{abstract}
%
\begin{IEEEkeywords}
Wordpress, Por2folios, Social Media, PPIV.
\end{IEEEkeywords}}
%%%%%%%%%%%%%%%%%%%%%%%%%%%%%%%%%%%%%%%%%%%%%%%%%%%%%%%%%%%%%%%%%%%%%%%%%%%%%%%%
% make the title area
\maketitle

\IEEEdisplaynontitleabstractindextext
\IEEEpeerreviewmaketitle
%%%%%%%%%%%%%%%%%%%%%%%%%%%%%%%%%%%%%%%%%%%%%%%%%%%%%%%%%%%%%%%%%%%%%%%%%%%%%%%%
%%%%%%%%%%%%%%%%%%%%%%%%%%%%%%%%%%%%%%%%%%%%%%%%%%%%%%%%%%%%%%%%%%%%%%%%%%%%%%%%
\section{Introduction}
% The very first letter is a 2 line initial drop letter followed
% by the rest of the first word in caps (small caps for compsoc).
% 
% form to use if the first word consists of a single letter:
% \IEEEPARstart{A}{demo} file is ....
% 
% Here we have the typical use of a "E" for an initial drop letter
% and "STE" in caps to complete the first word.
\IEEEPARstart {T}{his} project was commissioned as part of the PPIV course. Its main objective is to develop, document and report the development process of a platform to hold all the previous works and projects developed within the PPIV course. As such we documented all the requirements and had several meetings with Professor Rui Cruz, who gave us not only the necessary information we needed, as well as the analysis criteria and requirements specifications.

Todos os anos, dezenas de alunos dos \ac{MEIC} ingressam numa das duas Unidades Curriculares (UCs) de Portfolio Pessoal existentes. Estas unidades UCs permitem aos alunos desenvolver as suas soft-skills, capacidades não técnicas, transversais a qualquer area profissional, que são ferramentas essenciais para o desenvolvimento de todo o ser humano. Nesta Unidade Curricular (UC), os alunos escolhem uma das varias atividades disponíveis na plataforma da UC para desenvolver ao longo do semestre e, no fim do semestre letivo, submetem dois relatórios sobre as atividades desenvolvidas e as lições aprendidas com o decorrer da tarefa escolhida.

%%%%%%%%%%%%%%%%%%%%%%%%%%%%%%%%%%%%%%%%%%%%%%%%%%%%%%%%%%%%%%%%%%%%%%%%%%%%%%%%

\section{Meetings}

Durante o decorrer desta atividade, foram realizadas três reuniões envolvendo o promotor e orientador da atividade, o Professor Rui Santos Cruz, e os três alunos inscritos na atividade. Descreve-se de seguida os aspetos e decisões mais relevantes de cada reunião.

\subsection{Meeting 1}

The first meeting happened right at the beginning of the semester. This was one the the most important steps for this project. This was our first contact with the project, so to say, as up to this day, we had an idea on what the project would be, but we still had to hear the project owner's requirements and overall functionalities. During this meeting, some new key elements were set, as for example, our project would have its own dedicated Virtual Machine, and we would get root privileges over this machine, having the opportunity to build something from scratch, and to configure it as we wish.
	We also agreed on a layout for the front page of this project.
	
\subsection{Meeting 2}

This second meeting was also very important, as it happened during the end of April, beginning of May. A lot happened between meetings. For a start, we developed the first version of the project, and deployed it to our own private server. During this time, we got access to the final Virtual Machine, where we were able to install a server configuration tool, and to add an apache server and a database.
Having this work finished, we have discussed with our client the status of our project, where we had the opportunity to show the current status of the prototype, and also the changes on the final VM.Everything was approved, and we were sure to be on a good path to a successful project.

\subsection{Meeting 3}

	This final meeting was the shortest one, as it was an unscheduled one. We got to meet our colleague Nuno, and did integrate him on our platform. We have also discussed the next steps, and include Nuno on every decision and planning. Also, the new "tags" feature was discussed during this meeting.
	Regarding the server, we had the opportunity to show to the client the platform running already at its final server.
	As a final note, the only technical glitch to be solved at the end of this meeting was to connect the VM's IP to the final domain.
	(por2folios.tecnico.ulisboa.pt)

%%%%%%%%%%%%%%%%%%%%%%%%%%%%%%%%%%%%%%%%%%%%%%%%%%%%%%%%%%%%%%%%%%%%%%%%%%%%%%%%

\section{The Project}
The table of Scoring Rubrics (at the bottom of the first page of each Report) depends on the Report Type (ACTIVITY or LEARNINGS). Therefore, you must pay attention to have the correct table displayed when you select your Report Type in lines 21 or 22 of the \textbf{PP\_Report\_Preamble.tex} document. Failure to do so means that your Report WILL NOT BE ACCEPTED for evaluation.

%%%%%%%%%%%%%%%%%%%%%%%%%%%%%%%%%%%%%%%%%%%%%%%%%%%%%%%%%%%%%%%%%%%%%%%%%%%%%%%%

\section{Requirements}

Nesta secção iremos de forma resumida e elucidar como foi feito o levantamento dos requisitos da plataforma Por2folios. Esta secção contém informações importantes sobre as funcionalidades do protótipo a ser desenvolvido. Desta forma, serão abordadas definições iniciais dos requisitos do sistema que visa auxiliar o Professor, ou futuros alunos, no processo das características solicitadas para a plataforma. Pois, é de fundamental importância se obter uma boa compreensão das necessidades da plataforma Por2folios.

\subsection{Requisitos Funcionais}

Os requisitos funcionais do projeto são as funcionalidades que serão utilizadas diretamente no processo de solicitação de serviços, acompanhamento e aprovação dentre outros necessários à plataforma.

\subsubsection{Back-office Login}

Proporcionar aos utilizadores pré-registados pelo Administrador fazerem login através do back-office. Estar com a sessão iniciada é um factor fundamental para gerir a informação da plataforma.

\subsubsection{Aprovar Posts}

Aprovar posts neste caso é fundamental para que haja moderação do fluxo de informação da plataforma. Tal é moderado por utilizadores que tenham permissões para o fazer.

\subsubsection{Gerir Informação}

Toda a informação será gerida por utilizadores que tenham permissões para o fazer, sejam eles administradores ou apenas moderadores com a responsabilidade de gerir informação da página.

\subsection{Requisitos Não-Funcionais}

Estes requisitos visam ofertar e melhorar a qualidade do sistema. Neste sentido, haverá um esforço maior para que questões referentes à usabilidade e confiança do sistema sejam padronizadas e efetivamente adequadas ao serviço prestado.

%%%%%%%%%%%%%%%%%%%%%%%%%%%%%%%%%%%%%%%%%%%%%%%%%%%%%%%%%%%%%%%%%%%%%%%%%%%%%%%%

\section{Developments}

Esta secção descreve o desenvolvimento deste projeto ao longo do semestre, explicitando o fio condutor que leva desde a ideia inicial de criar uma plataforma com todos os relatórios ate resultados finais que conseguimos obter.

\subsection{CENAS}


%%%%%%%%%%%%%%%%%%%%%%%%%%%%%%%%%%%%%%%%%%%%%%%%%%%%%%%%%%%%%%%%%%%%%%%%%%%%%%%%
\section{Parte técnica}
For the Por2Folios platform implementation a fixed address machine was provided, hosted within the Virtual Machine of Instituto Superior Técnico. This machine was given to us, along with root accesses, with no restrictions of use whatsoever. The machine had 1Gb of Ram memory, 2Gb of virtual memory and 37Gb of storage capacity. The machine also had an out-of-date Ubunto distribution installed. To make the machine accessible we started by updating the Ubunto distribution to a more recent one (with Ubunto, the more recent the distribution, the more reliable and safer it is). Afterwards, we discussed what was the next step to take. Since we were working with a virtual machine to which we had no physical access, we decided to install a program to facilitate server management through the internet.  We used WebMin, to which access is made by the port 10000, enabling us to manage the server efficiently. This made server maintenance user friendly.

We had previously discussed what would be the best platform to develop Por2Folios. After exploring several \ac{CMS}, we concluded that WordPress was the one that better fitted our requirements. This was done before having access to the provided machine so, initially, we got the server running on a private machine. With everything installed in the private machine we launched an Alpha version of the platform, tested some functionalities, and we were able to perform some basic operations.
Having access and a port for server administration already installed in the provided machine, we then decided to migrate the WordPress from the private machine to the provided machine.
Before migrating WordPress (or any kind of \ac{CMS} actually) we had to install a HTTP server. Yet again, to facilitate the maintenance process, we used a tool which is largely documented and supported in the internet -- the Apache HTTP Server. This tool is free, runs in Windows and Unix, and 2.4 is the latest stable release. 
In order to store all the content, and to allow persistence associated with WordPress, we also had to implement a MySQL Database. Some changes were initially made to the MySql tables in order to integrate the DataBase with WordPress and, after migrating, some more changes were made to accommodate the change of domain associated with the migration to the Instituto Superior Técnico Domain.
After migrating, and to ensure security, an SSH Authentication access to the server using SSH-KEYS was implemented. This method becomes way more reliable by having the public passwords of users associated to the user account on the server. However, in order to facilitate access to future users, and to facilitate further maintenance, we continue to support the login with username and password. Both methods are currently supported.
At the moment the platform is in Beta status. Next semester we hope we'll continue to develop the platform, adding more functionalities and further enhancing the user experience.
%%%%%%%%%%%%%%%%%%%%%%%%%%%%%%%%%%%%%%%%%%%%%%%%%%%%%%%%%%%%%%%%%%%%%%%%%%%%%%%%

\subsection{Figures}
ADICINAR IMAGEM ? 
\ref{fig_sim}:

\begin{figure}[htb]
\centering
\includegraphics[width=1\linewidth]{soft_skills.png}
\caption{The Soft-Skills Tree}
\label{fig_sim}
\end{figure}

%%%%%%%%%%%%%%%%%%%%%%%%%%%%%%%%%%%%%%%%%%%%%%%%%%%%%%%%%%%%%%%%%%%%%%%%%%%%%%%%

\section{Future Work}

%%%%%%%%%%%%%%%%%%%%%%%%%%%%%%%%%%%%%%%%%%%%%%%%%%%%%%%%%%%%%%%%%%%%%%%%%%%%%%%%
\section{\IfLanguageName{english}{Conclusion}{Conclusão}}
The conclusions. Lorem ipsum dolor sit amet, consectetur adipiscing elit. Cras sed sapien quam. Sed dapibus est id enim facilisis, at posuere turpis adipiscing. Quisque sit amet dui dui.

Duis rhoncus velit nec est condimentum feugiat. Donec aliquam augue nec gravida lobortis. Nunc arcu mi, pretium quis dolor id, iaculis euismod ligula. Donec tincidunt gravida lacus eget lacinia. Lorem ipsum dolor sit amet, consectetur adipiscing elit.
%%%%%%%%%%%%%%%%%%%%%%%%%%%%%%%%%%%%%%%%%%%%%%%%%%%%%%%%%%%%%%%%%%%%%%%%%%%%%%%%
% references section
\bibliographystyle{IEEEtran}
%\bibliography{PP_Report_bib}
\bibliography{Mendeley}
%%%%%%%%%%%%%%%%%%%%%%%%%%%%%%%%%%%%%%%%%%%%%%%%%%%%%%%%%%%%%%%%%%%%%%%%%%%%%%%%
% biography section
% 
\begin{IEEEbiography}[{\includegraphics[width=1in,height=1.25in,clip,keepaspectratio]{bruno.png}}]{Bruno Cardoso}
Here I am. I am pursuing my Engineering studies at \ac{IST}. Starting my masters in Interaction and Visualization, and also on business systems. On my "spare" time, I am also developing a project at INESC-ID, at VIMMI.
\end{IEEEbiography}
\begin{IEEEbiography}
[{\includegraphics[width=1in,height=1.25in,clip,keepaspectratio]{francisco.png}}]{Francisco Maria Calisto}
I am pursuing my Information Systems and Computer  Engineering studies at \ac{IST}. I am also a VIMMI collaborator at INESC-ID. Currently I am working in a StartUp project called Agroop and I am Founder \& Front-end Developer of opprDev a oppr Group organisation.
\end{IEEEbiography}
\begin{IEEEbiography}
[{\includegraphics[width=1in,height=1.25in,clip,keepaspectratio]{me.png}}]{Nuno Sousa}
Here I am. I am pursuing my Engineering studies at \ac{IST}. Lorem ipsum dolor sit amet, consectetur adipiscing elit. Cras sed sapien quam. Sed dapibus est id enim facilisis, at posuere turpis adipiscing. Quisque sit amet dui dui.Lorem ipsum dolor sit amet, consectetur adipiscing elit. 
\end{IEEEbiography}

%%%%%%%%%%%%%%%%%%%%%%%%%%%%%%%%%%%%%%%%%%%%%%%%%%%%%%%%%%%%%%%%%%%%%%%%%%%%%%%%
% *** DEFINITION OF ACRONYMS ***
	\acrodef{CMS}{Content Management Systems}
	\acrodef{CPU}{Central Processing Unit}
	\acrodef{GUI}{Graphical User Interface}
	\acrodef{HTTP}{Hypertext Transfer Protocol}
	\acrodef{IST}{Instituto Superior Técnico}
	\acrodef{INESC-ID}{Instituto de Engenharia de Sistemas e Computadores - Investigação e Desenvolvimento}
	\acrodef{MEIC}{Mestrado em Engenharia Informática e de Computadores}
	\acrodef{VIMMI}{Visualization and Intelligent Multimodal Interfaces}
	\acrodef{LAN}{Local Area Network}
	\acrodef{PC}{Personal Computer}
	\acrodef{URL}{Uniform Resource Locator}
	\acrodef{VoD}{Video-on-demand}
	\acrodefplural{VoD}[VoDs]{Videos-on-demand}
	\acrodef{VoIP}{Voice over IP}
	\acrodef{WAN}{Wide Area Network}
	\acrodef{WLAN}{Wireless Local Area Network}
	\acrodef{WWAN}{Wireless Wide Area Network}
	\acrodef{WWW}{World Wide Web}

%%%%%%%%%%%%%%%%%%%%%%%%%%%%%%%%%%%%%%%%%%%%%%%%%%%%%%%%%%%%%%%%%%%%%%%%%%%%%%%%
\newpage
\onecolumn
%%%%%%%%%%%%%%%%%%%%%%%%%%%%%%%%%%%%%%%%%%%%%%%%%%%%%%%%%%%%%%%%%%%%%%%%%%%%%%%%
	
% that's all folks
\end{document}